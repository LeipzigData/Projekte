\documentclass[a4paper,11pt]{article}
\usepackage{a4wide,hyperref,ngerman}
\usepackage[utf8]{inputenc}

\parindent0pt
\parskip3pt
\setcounter{secnumdepth}{2}
\setcounter{tocdepth}{2}

\newenvironment{code}{\tt \begin{tabbing}
\hskip12pt\=\hskip12pt\=\hskip12pt\=\hskip12pt\=\hskip5cm\=\hskip5cm\=\kill}
{\end{tabbing}}
\def\ppw{{\char94\char94}}

\author{Marius Brunnert} 

\title{{\bf Datentransformationen im Projekt\\[1em]
    \emph{Gentrifizierungsprozesse in Leipzig}}\\[2em] Belegarbeit im
  Bachelor-Seminar Informatik,\\ Sommersemester 2014\\[2em]}

\date{22. September 2014}

\begin{document}
\begin{titlepage}
\maketitle
\end{titlepage}
\tableofcontents
\clearpage

\section{Einleitung}

Im Projekt „Gentrifizierungsprozesse in Leipzig“, das im Rahmen des Moduls
„Kreativität und Technik“ unter Leitung von Prof. Dr. Gräbe und in
Zusammenarbeit mit den Verantworlichen von \texttt{einundleipzig.de} (eine
Webseite, die sich mit diesem Thema auseinander setzt) statt gefunden hat,
haben wir uns in einer Gruppe von sechs Kursteilnehmern mit den
Gentrifizierungsprozessen in Leipzig befasst und uns zum Ziel gesetzt, eine
interaktive Karte zu erstellen, auf der diese visualisiert werden. Beim ersten
Treffen haben wir über unsere Vorkenntnisse und Erwartungen gesprochen, Ideen
gesammelt und schließlich das grundlegende Vorgehen geklärt. Um das Ziel zu
erreichen, haben wir uns an der Scrum-Methodik aus der Informatik orientiert,
allerdings „weekly“ statt „daily scrums“ geplant, und das Projekt in drei
Sprints gegliedert.

Ziel des ersten Sprints war zum einen, zu analysieren, welche Informationen
für unser Thema relevant sein könnten und mögliche Quellen zusammenzutragen,
die entsprechende Daten liefern können. Zum anderen sollten die Teammitglieder
sich kennen lernen, so dass im zweiten Sprint jeder eine konkrete Aufgabe, die
zu seinen Qualifikationen passt, übernehmen und ein funktionierender Workflow
etabliert werden konnte.

Im zweiten Sprint sollten dann die relevanten Daten aus den entsprechenden
Quellen gesammelt, in ein einheitliches Format gebracht und ins OntoWiki
transferiert werden. Außerdem sollte ein erster Prototyp der Karte erstellt
werden.

Der dritte und letzte Sprint hatte schließlich das Ziel, das zuvor Erarbeitete
zu einem „runden“ Gesamtergebnis zusammenzufassen, also die Daten, die sich
besonders für eine grafische Darstellung in einer Karte eignen, auf dieser
darzustellen.

Nach Abschluss des Projekts werden die Ergebnisse im Rahmen von
\texttt{einundleipzig.de} weiter genutzt. Außerdem habe ich mit den Daten noch
weitergearbeitet -- siehe Abschnitt \ref{Cube}.

\section{Meine Aufgaben}

\subsection{Erster Sprint}

Im ersten Sprint habe ich Kontakt mit Roman Grabolle und dem HausHalten
e.V. aufgenommen, um Informationen über Hausprojekte und alternative
Strukturen in Leipzig zu sammeln. Nach einigen kleineren Hürden habe ich auch
von beiden Rückmeldung bekommen und eine erste Übersicht über die verfügbaren
Informationen in der Gruppe vorgestellt. Da bei diesen Daten allerdings nie
ein Anspruch auf Vollständigkeit erhoben werden kann und die verschiedenen
Projekte in ihrer Bedeutung für die Gentrifizierung in Leipzig recht divergent
sind, haben wir uns letztlich entschlossen, diese Daten im Rahmen des Projekts
nicht zu verwenden. 

\subsection{Zweiter und dritter Sprint}

Im zweiten und dritten Sprint war ich zum einen die „Schnittstelle“
zwischen den Rohdaten und der Karte, das heißt, ich habe die Daten gesichtet,
geordnet, miteinander verknüpft (z.B. aus den Wohnungen pro Stadtteil und der
Größe des Stadtteils die Wohnungsdichte berechnet), sie ins .csv-Format
gebracht, falls sie nicht bereits in diesem vorlagen, und die .csv-Dateien
schließlich über das Repository zur Verfügung gestellt. Zum anderen habe ich
im OntoWiki entsprechende Ontologien erstellt und die Daten mit Hilfe von
.ttl-Dateien in das OntoWiki transferiert.  

\subsection{Nach dem Projekt}

Nach Abschluss des eigentlichen Projekts habe ich mich noch mit der Frage
beschäftigt, wie man die Daten des Leipzig-Informationssystems (LIS), die wir
zum Teil auch für unser Projekt genutzt haben, als \emph{RDF Data Cube}
darstellen kann, und schließlich einen solchen Data Cube mit Beispieldaten
erstellt.

\section{Darstellung der Daten im RDF}

\subsection{Allgemeine Form}

Die grundlegende Form unserer .ttl-Dateien sieht wie folgt aus:
\begin{code}
<prefixes>\\
<ldot:<ortsteil> gm:<attribut><yy> {\dq}<wert>{\dq}{\ppw}xsd:<datatype> .>*
\end{code}
Die \texttt{<prefixes>} sind in jeder .ttl-Datei identisch und definieren
Namensraum-Präfixe:
\begin{code}
@prefix owl:  <http://www.w3.org/2002/07/owl\#> .\\
@prefix rdf:  <http://www.w3.org/1999/02/22-rdf-syntax-ns\#> .\\
@prefix rdfs: <http://www.w3.org/2000/01/rdf-schema\#> .\\
@prefix xsd:  <http://www.w3.org/2001/XMLSchema\#> .\\
@prefix gm:   <http://leipzig-data.de/Data/Gentri-14/model/> .\\
@prefix ldot: <http://leipzig-data.de/Data/Ortsteil/> .
\end{code}
Für \texttt{<ortsteil>} wird der entsprechende Leipziger Ortsteil -- ohne
Umlaute -- eingetragen. Zum Beispiel „Zentrum-Ost“, „Schleussig“,
„Gruenau-Nord“ und so weiter.

Für \texttt{<attribut>} wird die gemessene Beobachtung benannt. Zum Beispiel
„Alter“, „Miete“, „IsAbwanderung“ und so weiter.

Für \texttt{<yy>} wird die auf zwei Ziffern abgekürzte Jahreszahl eingetragen.
Zum Beispiel „00“, „01“ und so weiter.

Für \texttt{<wert>} wird der beobachtete Wert angegeben, wobei
Nachkommastellen durch einen Punkt abgetrennt werden. Zum Beispiel „40“,
„6.23“, und so weiter.

Für \texttt{<datatype>} wird, je nach Beobachtung, entweder \texttt{decimal}
(mit Nachkommastellen) oder \texttt{integer} (ohne Nachkommastellen) gewählt.

Dieser Teil kann theoretisch beliebig oft mit unterschiedlich gewählten
Attributen vorkommen. In der Praxis sind es allerding $63 \cdot y$ Zeilen,
wobei 63 die Anzahl der Leipziger Ortsteile ist und $y$ für die Anzahl der
Jahre steht, für die Daten vorliegen.

\subsection{Ein konkretes Beispiel}

\begin{code}
@prefix owl:  <http://www.w3.org/2002/07/owl\#> .\\
@prefix rdf:  <http://www.w3.org/1999/02/22-rdf-syntax-ns\#> .\\
@prefix rdfs: <http://www.w3.org/2000/01/rdf-schema\#> .\\
@prefix xsd:  <http://www.w3.org/2001/XMLSchema\#> .\\
@prefix gm:   <http://leipzig-data.de/Data/Gentri-14/model/> .\\
@prefix ldot: <http://leipzig-data.de/Data/Ortsteil/> .\\[6pt]

ldot:Zentrum gm:alter00 {\dq}49.1{\dq}{\ppw}xsd:decimal .\\
ldot:Zentrum-Ost gm:alter00 {\dq}45.2{\dq}{\ppw}xsd:decimal .\\
ldot:Zentrum-Suedost gm:alter00 {\dq}45.1{\dq}{\ppw}xsd:decimal .\\
ldot:Zentrum-Sued gm:alter00 {\dq}45.5{\dq}{\ppw}xsd:decimal .\\
usw.
\end{code}

\section{Die Matrix} \label{matrix}

\subsection  {Allgemeines}
Da wir unsere Karte nach Ortsteilen gegliedert haben, einige uns vorliegende
Daten (Informationen zu Wohnorten Leipziger Studenten) allerdings nach
Postleitzahlen gruppiert waren, war eine Umrechnung nötig, für die ich eine
Matrix erstellt habe. Leider kann mit dieser Matrix nur eine näherungsweise
Umrechnung erfolgen, da es mehrere Ungenauigkeiten gibt: 
\begin{enumerate}
\item Wenn ein Postleitzahlbereich mehrere Ortsteile (ganz oder zum Teil)
  abdeckt, was relativ häufig vorkommt, gibt es keine konkreten Informationen
  darüber, wie viele Einwohner, die unter dieser Postleitzahl gemeldet sind, in
  Ortsteil A und wie viele in Ortsteil B (und ggf.\ in Ortsteil C) leben.  Hier
  kann die Umrechnung also über einen zu bestimmenden \emph{Anteilsfaktor} nur
  näherungsweise erfolgen.

\item Zur Bestimmung eines solchen Anteilsfaktors bin ich wie folgt
  vorgegangen: Grundlage waren zwei
  Karten\footnote{\url{http://www.infos-sachsen.de/Orte/Stadt-L.php}}
  (Gliederung der Stadt nach PLZ bzw.\ nach Ortsteilen), aus denen die
  Zuordnung von Ortsteilen zu Postleitzahlbereichen abgelesen bzw.\ ein
  flächenmäßiger Anteil geschätzt wurde.  Diese Zahlen wurden mit den
  Bevölkerungszahlen der Stadtteile (Stand 2013) multipliziert und daraus die
  Anteilsfaktoren bestimmt.
\end{enumerate}
Das Ergebnis ist eine Matrix mit 34 Zeilen (entspricht den
Postleitzahlbereichen) und 63 Spalten (entspricht den Ortsteilen). In jeder
Position dieser Matrix ist angegeben, mit welchem Anteilsfaktor der Wert aus
einem Postleitzahlbereich $Y$ zu multiplizieren ist, um den Anteil dieses Werts
zu bestimmen, der auf den Stadtteil $X$ entfällt. Da sich ein
Postleitzahlbereich über maximal drei Ortsteile erstreckt, enthalten die
meisten Positionen eine null.  Nach Konstruktion summieren die Anteilsfaktoren
jeder Zeile zu eins.

\subsection{Beispielrechnungen}
\begin{enumerate}
\item Die Postleitzahl 04103 deckt die Ortsteile 01 und 02 zu jeweils 100\%
  ab.  Ortsteil 01 hatte (im Jahr 2013) 3\,980 Einwohner, Ortsteil 02 hatte
  11\,515 Einwohner.  Dies ergibt eine Gesamtzahl von 15\,495 Einwohnern, die
  unter der Postleitzahl 04103 zusammengefasst werden. Daran hat Ortsteil 01
  einen Anteil von $\frac{3980}{15495}\approx 25.7\%$ und Ortsteil 02 einen
  Anteil von $\frac{11515}{15495}\approx 74,3\%$. Die Anteilsfaktoren sind
  also 0.257 bzw.\ 0.743.

\item Die Postleitzahl 04159 deckt die Ortsteile 80 und 81 zu jeweils 100\%,
  sowie den Ortsteil 82 zu etwa 50\% ab. Ortsteil 80 hatte 13\,172 Einwohner,
  Ortsteil 81 hatte 6\,536 Einwohner, Ortsteil 82 hatte 3\,945 Einwohner -- da
  dieser allerdings nur zur Hälfte in der Postleitzahl 04159 enthalten ist,
  rechnen wir auch nur mit der halben Einwohnerzahl: 1\,972.5. Insgesamt deckt
  die Postleitzahl somit 21\,680.5 Einwohner ab. Daran hat Ortsteil 80 einen
  Anteil von $\frac{13172}{21680.5}\approx 60.8\%$, Ortsteil 81 einen Anteil
  von $\frac{6536}{21680.5}\approx 30.1\%$ und Ortsteil 82 einen Anteil von
  $\frac{6536}{21680.5}\approx 9.1\%$. Die Anteilsfaktoren sind also 0.608,
  0.301 und 0.091.
\end{enumerate}

\section{Darstellung der Daten als RDF Data Cube}\label{Cube}

\subsection{Die Prefixes} \label{Prefixes}
 
\begin{code}
@prefix rdf:            <http://www.w3.org/1999/02/22-rdf-syntax-ns\#> .\\
@prefix rdfs:           <http://www.w3.org/2000/01/rdf-schema\#> .\\
@prefix owl:            <http://www.w3.org/2002/07/owl\#> .\\
@prefix xsd:            <http://www.w3.org/2001/XMLSchema\#> .\\
@prefix dct:             <http://purl.org/dc/terms/> .\\
@prefix qb:             <http://purl.org/linked-data/cube\#> .\\
@prefix sdmx:           <http://purl.org/linked-data/sdmx\#> .\\
@prefix sdmx-concept:   <http://purl.org/linked-data/sdmx/2009/concept\#> .\\
@prefix sdmx-dimension: <http://purl.org/linked-data/sdmx/2009/dimension\#> .\\
@prefix sdmx-attribute: <http://purl.org/linked-data/sdmx/2009/attribute\#> .\\
@prefix sdmx-measure:   <http://purl.org/linked-data/sdmx/2009/measure\#> .\\
@prefix gcube:          <http://leipzig-data.de/Data/Gentri-14/cube/> .\\
@prefix ldot:           <http://leipzig-data.de/Data/Ortsteil/> .
\end{code}

\subsection{Das DataSet}

\begin{code}
gcube:GentriLeipzig a qb:DataSet ;\+\\
    rdfs:label {\dq}Gentrifizierung in Leipzig{\dq} ;\\
    dct:source <http://statistik.leipzig.de/statdist/index.aspx> ;\\
    qb:structure gcube:GentriDSD .
\end{code}
          
\subsection{Die DataStructureDefinition} \label{DSS}

\subsubsection{Allgemein}
\begin{code}
gcube:GentriDSD a qb:DataStructureDefinition ;\+\\
    qb:component \+\\{}
        [ qb:dimension gcube:district; ],\\{}
        [ qb:dimension gcube:year; ],\\{}
        [ qb:dimension qb:measureType; ],\\{}
        [ qb:attribute sdmx-attribute:unitMeasure; ],\\{}
        [ qb:measure gcube:<measure>; ]* .
\end{code}
        
Für \texttt{[qb:measure gcube:<measure>; ]*} müssen die verwendeten
\texttt{Measure}s eingesetzt werden. Es muss mindestens eine \texttt{Measure}
geben.
\subsubsection{Beispiel}
\begin{code}
\>\>  [ qb:measure gcube:extImmigrants; ],\\{}
\>\>  [ qb:measure gcube:meanRent; ],\\{}
        usw.
\end{code}

\subsection{Die Dimension(Propertie)s}

\begin{code}
gcube:district a rdf:Property, qb:DimensionProperty ;\+\\
    rdfs:label {\dq}Ortsteil{\dq}@de ;\\
    rdfs:subPropertyOf sdmx-dimension:refArea ;\\
    qb:concept sdmx-concept:refArea .
\end{code}
\begin{code}
gcube:year a rdf:Property, qb:DimensionProperty ;\+\\
    rdfs:label {\dq}Jahr{\dq}@de ;\\
    rdfs:subPropertyOf sdmx-dimension:refPeriod ;\\
    qb:concept sdmx-concept:refPeriod .\\
\end{code}

\subsection{Die Measure(Propertie)s} \label{Measures}

\subsubsection{Allgemein}
\begin{code}           
gcube:<measure> a rdf:Property, qb:MeasureProperty ;\+\\
    rdfs:label {\dq}<measureLabel>{\dq}@de ;\\
    rdfs:subPropertyOf sdmx-measure:obsValue ;\\
    dct:source <source> ;\\
    rdfs:range xsd:<range> .
\end{code}

Für \texttt{<measure>} wird der Name der \texttt{MeasureProperty} angegeben,
für \texttt{<measureLabel>} eine genauere Bezeichnung (auf Deutsch), für
\texttt{<source>} ein Verweis auf die Quelle (in den folgenden Beispielen
weggelassen), aus der die Daten stammen, und für \texttt{<range>} der erlaubte
Wertebereich (hier: \texttt{integer} oder \texttt{decimal}).

\subsubsection{Beispiel}
\begin{code}
gcube:extEmigrants a rdf:Property, qb:MeasureProperty ;\+\\
    rdfs:label {\dq}Wegzüge nach Außerhalb{\dq}@de ;\\
    rdfs:subPropertyOf sdmx-measure:obsValue ;\\
    rdfs:range xsd:integer .
\end{code}
\begin{code}
gcube:meanAge a rdf:Property, qb:MeasureProperty ;\+\\
    rdfs:label {\dq}Durchschnittsalter{\dq}@de ;\\
    rdfs:subPropertyOf sdmx-measure:obsValue ;\\
    rdfs:range xsd:decimal .
\end{code}

\subsection{Die Attribute(Propertie)s} 
\label{Attributes}

\subsubsection{Allgemein}
\begin{code}
gcube:<attribute> a rdf:Property, qb:AttributeProperty ;\+\\
    rdfs:label {\dq}<attributeLabel>{\dq}@de .
\end{code}

Für \texttt{<attribute>} wird der Name der \texttt{AttributeProperty}
angegeben, für \texttt{<attributeLabel>} eine genauere Bezeichnung (auf
Deutsch).

\subsubsection{Beispiel}
\begin{code}
gcube:people a rdf:Property, qb:AttributeProperty ;\+\\
    rdfs:label {\dq}Menschen{\dq}@de .
\end{code}
\begin{code}
gcube:euro a rdf:Property, qb:AttributeProperty ;\+\\
    rdfs:label {\dq}Euro{\dq}@de .
\end{code}

\subsection{Die Observations} \label{Observations}

\subsubsection{Allgemein}
\begin{code}
gcube:O<year>-<dnum>-GR<ln> gcube:<measure> {\dq}<amount>{\dq}{\ppw}xsd:<datatype> ;\+\\
    qb:measureType gcube:<measure> ;\\
    gcube:district ldot:<dname> ;\\
    gcube:year {\dq}<year>{\dq}{\ppw}xsd:gYear ;\\
    sdmx-attribute:unitMeasure gcube:<attribute> ;\\
    a qb:Observation ;\\
    qb:dataset gcube:GentriLeipzig .
\end{code}
Für \texttt{<year>} wird das Jahr, auf das sich die Daten der Observation
beziehen, angegeben, für \texttt{<dname>} der Name (ohne Umlaute, „ss“ statt
„ß“, „ae“ statt „ä“ usw.) des entsprechenden Ortsteils, für \texttt{<ln>} eine
laufende Nummer über den unterschiedlichen Arten von \texttt{Observations}
bzw. \texttt{MeasureProperties} (\texttt{extimmigrants} hat Nummer 0,
\texttt{extemigrants} die Nummer 1, \texttt{locimmigrants} die Nummer 2,
usw.\ -- eine genaue Auf"|listung ist im Abschnitt \ref{Daten}) zu finden, für
\texttt{<measure>} der Name der entsprechenden \texttt{MeasureProperty} (also
die Art der \texttt{Observation}), für \texttt{<amount>} der beobachtete Wert,
für \texttt{<datatype>} der erlaubte Datentyp (hier entweder \texttt{integer}
oder \texttt{decimal}) und für \texttt{<attribute>} die entsprechende
\texttt{AttributeProperty} (also die Angabe, in welcher Einheit die Observation
„gezählt“ wird, z.B. „people“, „years“, oder „euro“).

\subsubsection{Beispiel}
\begin{code}
gcube:O2000-00-GR0 gcube:extImmigrants {\dq}208{\dq}{\ppw}xsd:integer ;\+\\
    qb:measureType gcube:extImmigrants ;\\
    gcube:district gcube:district00 ;\\
    gcube:year {\dq}2000{\dq}{\ppw}xsd:gYear ;\\
    sdmx-attribute:unitMeasure gcube:people ;\\
    a qb:Observation ;\\
    qb:dataset gcube:GentriLeipzig .
\end{code}
\begin{code}
gcube:O2000-01-GR0 gcube:extImmigrants {\dq}274{\dq}{\ppw}xsd:integer ;\+\\
    qb:measureType gcube:extImmigrants ;\\
    gcube:district gcube:district01 ;\\
    gcube:year {\dq}2000{\dq}{\ppw}xsd:gYear ;\\
    sdmx-attribute:unitMeasure gcube:people ;\\
    a qb:Observation ;\\
    qb:dataset gcube:GentriLeipzig .
\end{code}

\subsection {Bisher vorhandene Daten} \label {Daten}
\begin{description}
\setlength{\itemsep}{0pt}
\item[GR0:] Zuzüge in Ortsteil $x$ von außerhalb Leipzigs\\
  (\texttt{qb:measureType} \texttt{gcube:extImmigrants})
\item[GR1:] Wegzüge aus Ortsteil $x$ nach außerhalb Leipzigs\\
  (\texttt{qb:measureType} \texttt{gcube:extEmigrants})
\item[GR2:] Zuzüge in Ortsteil $x$ aus einem anderen Ortsteil\\
  (\texttt{qb:measureType} \texttt{gcube:locImmigrants})
\item[GR3:] Wegzüge aus Ortsteil $x$ in einen anderen Ortsteil\\
  (\texttt{qb:measureType} \texttt{gcube:locEmigrants})
\item[GR4:] Durchschnittsalter in Ortsteil $x$ (\texttt{qb:measureType}
  \texttt{gcube:meanAge})
\item[GR5:] Studentenanteil in Ortsteil $x$ (\texttt{qb:measureType}
  \texttt{gcube:studPercent})
\item[GR6:] Durchschnittsmiete in Ortsteil $x$ (\texttt{qb:measureType}
  \texttt{gcube:meanRent})
\item[GR7:] Wohnungsdichte in Ortsteil $x$ (\texttt{qb:measureType}
  \texttt{gcube:aptDensity})
\item[GR8:] Arbeitslose in Ortsteil $x$ (\texttt{qb:measureType}
  \texttt{gcube:unemployeds})
\end{description}
\subsection{Probleme} 
\label{fehler}

\paragraph{Language-Tags:} 
Wenn die .ttl-Dateien Language-Tags (\texttt{@de}) enthalten, kommt es im
OntoWiki aus bisher nicht nachvollziehbaren Gründen zu Fehlermeldungen.  Ein
Beispiel hierfür ist in der Datei \texttt{SPARQL\_ERROR.txt} zusammengestellt.

\paragraph{qb-components:} 
Wenn man die \texttt{qb-components} als Blank Nodes definiert, gehen beim
Transfer ins OntoWiki die Meta- und Kontextinformationen der Komponenten
verloren. Um das zu vermeiden, wurde auch eine Variante ohne Blank Nodes
erstellt. (Siehe Abschnitt \ref{noblank})

\section{Weitere Schritte}

\subsection{Automatisierung der Umwandlung von .csv zu .ttl}
\label{Umwandlung}
\subsubsection{Allgemeines}

Da die Daten im LIS recht einheitlich im .csv-Format vorliegen, erscheint es
sinnvoll, die Übertragung der weiteren LIS-Daten in den Data-Cube so weit wie
möglich zu automatisieren. Ausgehend von einer entsprechenden .csv-Datei lässt
sich eine weitgehend automatische Umwandlung in eine
„Ergänzungs-.ttl-Datei“ in folgenden 8 Teilschritten umsetzen: 
\begin{enumerate}
\item Erstellen der Prefixes. Die Prefixes sind (in diesem Cube) immer
  identisch, müssen aber in jeder RDF-Datei, wie unter \ref{Prefixes}
  beschrieben, angegeben werden. 
\item Am \texttt{DataSet} ändert sich nichts, dieses muss also bei Ergänzungen
  nicht noch einmal aufgeführt werden.
\item In der \texttt{DataStructureDefinition muss} die neue \texttt{Measure}
  mit Namen als solche definiert und als \texttt{qb:component} angeben
  werden.  \label{newmeasure}
\item An den \texttt{DimensionProperties} ändert sich nichts, diese müssen
  also bei Ergänzungen nicht noch einmal aufgeführt werden.
\item Die neue \texttt{Measure} muss, wie unter \ref{Measures} angegeben, als
  \texttt{MeasureProperty} definiert werden. Die \texttt{<source>}-Angabe kann
  dabei nicht direkt aus der .csv-Datei bestimmt werden.
\item Entweder muss der neuen \texttt{Measure} eine passende
  \texttt{AttributeProperty} zugeordnet werden (in Schritt \ref{newobs}) oder,
  wie unter \ref{Attributes} angegeben, eine neue erstellt werden. Dieser
  Schritt kann nicht voll automatisch erfolgen.
\item An den \texttt{Districts} ändert sich nichts, diese müssen also bei
  Ergänzungen nicht noch einmal aufgeführt werden.
\item Für jede Zelle in der .csv-Datei muss eine \texttt{Observation}, wie
  unter \ref{Observations} angegeben, erstellt werden.\label{newobs}
\end{enumerate}

\subsubsection {Daten aktualisieren}
Um aktuellere Versionen der Daten einzupflegen, zu denen es bereits
\texttt{Measure}s gibt, muss man lediglich neue \texttt{Observation}s mit den
passenden Parametern erstellen.

\subsubsection{Besonderheiten}

In einigen .csv-Dateien des LIS fehlen Angaben für einige Ortsteile, bzw. sind
z.\,T.\ die Angaben für mehrere Ortsteile in einem Ortsteil zusammengefasst.
In der .csv-Datei tauchen in den fehlenden oder zusammengefassten Zellen nur
\dq.\dq-Zeichen auf. (Zum Beispiel werden beim Wohnungsbestand vor 2010 die
Ortsteile 27, 28 und 29, sowie die Ortsteile 54 und 55 zusammengefasst.)
Sollten solche Zeichen in der .csv-Datei vorkommen, ist es also wieder nötig,
manuell einzugreifen und direkt ins LIS zu schauen, wo durch Annotationen
darauf hingewiesen wird, welcher Zusammenhang konkret besteht. Fehlende
Informationen für einzelne Zellen habe ich bisher mit dem Wert „-1“
gekennzeichnet.  Sollte in anderen Statistiken auch mit negativen Werten
gearbeitet werden, müsste man da nach Alternativen suchen.

\subsection{Beispiel-Turtle-Datei zur Erweiterung des bestehenden Cubes}

\begin{code}
@prefix rdf:            <http://www.w3.org/1999/02/22-rdf-syntax-ns\#> .\\
@prefix rdfs:           <http://www.w3.org/2000/01/rdf-schema\#> .\\
@prefix owl:            <http://www.w3.org/2002/07/owl\#> .\\
@prefix xsd:            <http://www.w3.org/2001/XMLSchema\#> .\\
@prefix dct:             <http://purl.org/dc/terms/> .\\
@prefix qb:             <http://purl.org/linked-data/cube\#> .\\
@prefix sdmx:           <http://purl.org/linked-data/sdmx\#> .\\
@prefix sdmx-concept:   <http://purl.org/linked-data/sdmx/2009/concept\#> .\\
@prefix sdmx-dimension: <http://purl.org/linked-data/sdmx/2009/dimension\#> .\\
@prefix sdmx-attribute: <http://purl.org/linked-data/sdmx/2009/attribute\#> .\\
@prefix sdmx-measure:   <http://purl.org/linked-data/sdmx/2009/measure\#> .\\
@prefix gcube:          <http://leipzig-data.de/Data/Gentri-14/cube/> .\\[6pt]

gcube:DataSetDef qb:component [ qb:measure gcube:aptDensity ] .\\[6pt]

gcube:aptDensity a rdf:Property, qb:MeasureProperty ;\+\\
    rdfs:label {\dq}Wohnungsdichte{\dq}@de ;\\
    rdfs:subPropertyOf sdmx-measure:obsValue ;\\
    rdfs:range xsd:integer .\-\\[6pt]

gcube:aptsPerKm a rdf:Property, qb:AttributeProperty ;\+\\
    rdfs:label {\dq}Wohnungen pro Quadratkilometer{\dq}@de .\-\\[6pt]

gcube:O2000-00-GR2 gcube:aptDensity {\dq}2209{\dq}{\ppw}xsd:integer ;\+\\
    qb:measureType gcube:aptDensity ;\\
    gcube:district gcube:district00 ;\\
    gcube:year {\dq}2000{\dq}{\ppw}xsd:gYear ;\\
    sdmx-attribute:unitMeasure gcube:aptsPerKm ;\\
    a qb:Observation ;\\
    qb:dataset gcube:GentriLeipzig .\-\\[6pt]

gcube:O2000-01-GR2 gcube:aptDensity {\dq}1462{\dq}{\ppw}xsd:integer ;\+\\
    qb:measureType gcube:aptDensity ;\\
    gcube:district gcube:district01 ;\\
    gcube:year {\dq}2000{\dq}{\ppw}xsd:gYear ;\\
    sdmx-attribute:unitMeasure gcube:aptsPerKm ;\\
    a qb:Observation ;\\
    qb:dataset gcube:GentriLeipzig .\-\\[6pt]

usw.
\end{code}

\section{Anhang}

\subsection{Cube-Variante ohne Blank Nodes}\label{noblank}

\subsubsection{Der Cube}

Am \texttt{Cube} selbst muss lediglich in der \texttt{DataStructureDefinition}
geändert werden, wie die Komponenten angegeben werden. Statt die Komponenten
wie unter \ref{DSS} angegeben als Blank Nodes zu definieren, wird jeder
benötigten Komponente eine konkrete URI zugewiesen, die dann separat definiert
werden muss.

\begin{code}
gcube:datasetdef a qb:DataStructureDefinition ;\+\\
    qb:component\+\\
        gcube:districtDim,
        gcube:yearDim,
        gcube:measureDim,\\
        gcube:attributeComp,
        gcube:measureExtImm,
        gcube:measureExtEm,
        <\ldots> .\-\-\\[6pt]
                 
gcube:districtDim      qb:dimension gcube:district .\\
gcube:yearDim          qb:dimension gcube:year .\\
gcube:measureDim       qb:dimension qb:measureType .\\
gcube:attributeComp    qb:attribute sdmx-attribute:unitMeasure .\\
gcube:measureExtImm    qb:measure   gcube:extImmigrants .\\
gcube:measureExtEm     qb:measure   gcube:extEmigrants .
<\ldots> .
\end{code}

Für \texttt{<\ldots>} würden weitere Measures benannt und definiert werden.

Ansonsten ändert sich am Cube nichts.

\subsubsection{Die Erweiterung}

Auch bei der Erweiterung des Cubes muss das selbstverständlich entsprechend
beachtet werden. In Schritt \ref{newmeasure} der Umwandlung (\ref{Umwandlung})
muss die neue \texttt{Measure} zunächst in der
\texttt{DataStruc\-ture\-Definition} mit einer URI benannt und dann separat
definiert werden.

\begin{code}
gcube:datasetdef qb:component gcube:measure\_aptdens .\\
gcube:measure\_aptdens  qb:measure gcube:aptdensity .
\end{code}

\subsection{Dateien}

\subsubsection{RDF-Daten für die GentriMap}

Verzeichnis \texttt{SimpleRDF}:
\begin{itemize}
\setlength{\itemsep}{-2pt}
\item \texttt{alter00\_13.ttl} -- Durchschnittsalter in den Ortsteilen von 2000
  bis 2013.
\item \texttt{anteilarbeitslose00\_13.ttl} -- Arbeitslosenanteil in den
  Ortsteilen von 2000 bis 2013
\item \texttt{aszuwanderung00\_13.ttl} -- Zuwanderung von außerhalb Leipzig in
  die Ortsteile von 2000 bis 2013
\item \texttt{isabwanderung00\_13.ttl} -- Abwanderung von den Ortsteilen in
  einen anderen Ortsteil von 2000 bis 2013
\item \texttt{iszuwanderung00\_13.ttl} -- Zuwanderung in die Ortsteile aus
  einem anderen Ortsteil von 2000 bis 2013
\item \texttt{kinderarmut07+09+11.ttl} -- Prozentuale Kinderarmut in den
  Ortsteilen in 2007, 2009 und 2011 (Nicht exakt!)
\item \texttt{studentenanteilot13.ttl} -- Studentenanteil in den Ortsteilen in
  2013\footnote{Wie viel Prozent der Einwohner des Ortsteils sind Studenten?}
\item \texttt{studentenanteilst13.ttl} -- prozentualer Anteil an Leipziger
  Studenten in den Ortsteilen in 2013\footnote{Wie viel Prozent der Studenten
    Leipzigs leben im Ortsteil?}
\item \texttt{wohnungsdichte00\_11.ttl} -- Wohnungen pro km$^2$ in den
  Ortsteilen von 2000 bis 2011
\end{itemize}

\subsubsection{Der RDF Data Cube}

Diese Daten sind im Verzeichnis \texttt{CUBE} zu finden.

DataSet, DataStructureDefinition, Dimension-, Measure- und
AttributeProperties:
\begin{itemize}
\setlength{\itemsep}{-2pt}
\item \texttt{cubeBaseBlank.ttl} -- mit BlankNodes, ohne LanguageTags
\item \texttt{cubeBaseBlankLT.ttl} -- mit BlankNodes, mit LanguageTags
\item \texttt{cubeBaseNoBlank.ttl} -- ohne BlankNodes, ohne LanguageTags
\item \texttt{cubeBaseNoBlankLT.ttl} -- ohne BlankNodes, mit LanguageTags
\end{itemize}

Die 63 Leipziger Ortsteile als \texttt{gcube:district}:
\begin{itemize}
\setlength{\itemsep}{-2pt}
\item \texttt{districts.ttl} -- ohne LanguageTags
\item \texttt{districtsLT.ttl} -- mit LanguageTags
\end{itemize}

Die Observations (aufgeschlüsselt nach Ortsteilen):
\begin{itemize}
\setlength{\itemsep}{-2pt}
\item \texttt{observations/obsaszz.ttl} -- GR0: Zuzüge nach Leipzig
\item \texttt{observations/obsaswz.ttl} -- GR1: Wegzüge aus Leipzig 
\item \texttt{observations/obsiszz.ttl} -- GR2: Zuzüge innerhalb Leipzigs
\item \texttt{observations/obsiswz.ttl} -- GR3: Wegzüge innerhalb Leipzigs 
\item \texttt{observations/obsage.ttl} -- GR4:  Durchschnittsalter
\item \texttt{observations/obsstuda.ttl} -- GR5: Studentenanteil
\item \texttt{observations/obsrent.ttl} -- GR6: Durchschnittsmiete
\item \texttt{observations/obsapt.ttl} -- GR7: Wohnungsdichte
\item \texttt{observations/obsunemp.ttl} -- GR8: Arbeitslose
\end{itemize}

\subsubsection{Sonstiges}

\begin{itemize}
\setlength{\itemsep}{-2pt}
\item \texttt{/matrixPLZtoORTSTEIL.csv} -- Die unter \ref{matrix} beschriebene
  Matrix mit einer zusätzlichen Spalte (Angabe der PLZ) und einer zusätzlichen
  Zeile (Angabe des Ortsteils).
\item \texttt{/SPARQL\_ Error.txt} -- Ein Beispiel für die im Abschnitt
  \ref{fehler} genannten SPARQL-Fehler\-meldungen.
\item \texttt{/Projektbericht.tex} -- Dieser Projektbericht als \LaTeX-Datei.
\end{itemize}

\end{document}
