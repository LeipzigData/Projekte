\documentclass[a4paper,11pt]{article}
\usepackage{a4wide,german,amsmath,url}
\usepackage[utf8]{inputenc}
\parindent0cm
\parskip3pt

\title{Bericht zum Energie-Projekt: Solarstromprognose\\ Wintersemester
  2013/14}

\author{Ramon Bernert}
\date{Januar 2014}

\begin{document}
\maketitle

\section{Einleitung} 
Die EEG-Stammdaten enthielten zu jeder Anlage die Eigenschaft „Leistung“.
Diese Kennzahl bezeichnet die installierte bzw.\ maximale elektrische
Leistung, die durch die entsprechende Anlage unter optimalen Bedingungen
geliefert werden kann. Zu diesen optimalen Bedingungen gehören insbesondere
bei den erneuerbaren Energien der Standort und die Wettereinflüsse. Da im
normalen Betrieb diese Bedingungen nie bzw.\ sehr selten erfüllt sind, weicht
die tatsächlich eingespeiste Leistung zum Teil erheblich von der maximal
möglichen Leistung ab. Da die Vergütung der Anlagenbetreiber über die wirklich
eingespeiste Leistung berechnet wird und die Netzbetreiber die Menge der ein-
und ausgespeisten Energie im Gleichgewicht halten müssen, um die
Stromversorgungsnetze stabil zu halten, haben diese ein besonderes Interesse,
die eingespeiste Leistung zu prognostizieren, damit evtl.\ auftretende Über-
bzw.\ Unterproduktion am Besten schon im Voraus eingeplant und durch
entsprechende Regelenergie abgefangen werden kann. Deshalb wurden für die
verschiedenen Energieträger zahlreiche Modelle entwickelt, um mit Hilfe von
Wetterprognosen diese Leistung vorherzusagen.

Auch innerhalb unserer Projektgruppe kam nach den ersten Treffen recht schnell
die Frage auf, wie viel elektrische Leistung eine Anlage in Zukunft einspeisen
wird. Da die innerhalb unseres Projekts vorliegenden Daten zu einem großen
Teil Daten zu Photovoltaikanlagen enthielten, beschränke ich mich bei den
weiteren Ausführungen auf Prognosen für den Solarstrom.

\section{Solarstromprognose}
\subsection{Allgemein}
Die Solarstromprognose bezeichnet die Vorhersage der elektrischen Leistung
einer Photovoltaikanlage (PV-Anlage) im Verlauf der nächsten Tage. Die
wichtigsten Einflussfaktoren sind die eintreffende Strahlung der Sonne auf die
Erde sowie die Aufstellungsrichtung, der Neigungswinkel und die Art der
PV-Anlage.  Mittlerweile haben sich viele Anbieter von Solarstromprognosen
innerhalb der Energiewirtschaft etabliert, welche aber die zugrunde liegenden
Modelle weder veröffentlichen oder gar dokumentieren. Das Institute of Energy
and Transport (IET) der Europäische Kommission stellt jedoch das
\emph{Photovoltaic Geographical Information System} (PVGIS) öffentlich zur
Verfügung. Dieses System ermöglicht durch Eingabe der oben genannten Parameter
eine einfache monatliche Leistungsprognose. Dabei werden die ablaufenden
Berechnungen zu einem Großteil gut dokumentiert dargestellt.

\subsection{Photovoltaic Geographical Information System (PVGIS)}
Das grundlegende mathematische Modell des PVGIS beruht auf folgender Formel,
wobei $P$ die tatsächlich abgegebene Leistung bezeichnet:
\begin{gather*}
  P=\frac{G(\alpha,\beta)}{1000}\, P_{pk}\,
  \mathit{eff}_{rel}(G(\alpha,\beta),T_m)\,.
\end{gather*}
Dabei steht $G(\alpha,\beta)$ für die Globalstrahlung der Sonne unter
Berücksichtigung des Azimuts $\alpha$ und des Neigungswinkels der
Aufstellfläche $\beta$. Der Azimut ist dabei der nach der Himmelsrichtung
orientierte Horizontalwinkel. Die Globalstrahlung setzt sich aus folgenden
Komponenten zusammen:
\begin{gather*}
  G(\alpha,\beta)=I(\alpha,\beta)+D(\alpha,\beta)+R(\alpha,\beta)\,.
\end{gather*}
Die Direktstrahlung $I(\alpha,\beta)$ bezeichnet die direkt auf die
Erdoberfläche auftreffende Sonnenstrahlung, während die Diffusstrahlung
$D(\alpha,\beta)$ jegliche Sonnenstrahlung beinhaltet, welche durch die
Erdatmosphäre diffus gestreut wird. $R(\alpha,\beta)$ fasst die reflektierte
Strahlung von der Erdoberfläche zusammen.

Die Kenngröße $P_{pk}$ beschreibt die im Photovoltaik-Bereich übliche, aber
nicht normgerechte, abgegebene elektrische Leistung unter
Standard-Testbedingungen.  Diese Angabe wird üblicherweise bei jeder PV-Anlage
genannt und ist abhängig von der Fläche und dem Wirkungsgrad des Solarmoduls.
Die Standard-Testbedingungen umfassen die Zelltemperatur
($25^\circ\mathrm{C}$), die Bestrahlungsstärke $1000\,\frac{W}{m^2}$ und das
Sonnenlichtspektrum $AM=1.5$.

Die Einflussgröße $\mathit{eff}_{rel}(G(\alpha,\beta),T_m)$ bezeichnet den
relativen Wirkungsgrad, welchervon der Umgebungstemperatur und der verwendeten
PV-Technologie abhängt. Mathematisch wird diese Größe wie folgt beschrieben:
\begin{align*}
  \mathit{eff}_{rel}(G(\alpha,\beta),T_m) &= l +k_1\,\ln(G') +k_2\,\ln(G')^2+
  k_3\,T_m +k_4\,T_m\,\ln(G')\\ &\qquad +k_5\,T_m\,\ln(G')^2 +k_6\, T_m^2\,.
\end{align*}
Dabei fließt in $T_m$ die Umgebungstemperatur $T_{amb}$, die Globalstrahlung
$G(\alpha,\beta)$ und die Art der Errichtung $k_T$ ein:
\begin{gather*}
  T_m= T_{amb}+k_T\,G(\alpha,\beta)-25\,\mathrm{{^\circ}C}\,.
\end{gather*}
Der Koeffizient $k_T$ beträgt bei einer freistehenden Anlage
$k_T=0.035\frac{{^\circ}C}{W\,m^2}$ und bei der Errichtung auf einem Dach
$k_T=0.05\frac{{^\circ}C}{W\,m^2}$.  Dieser Unterschied entsteht durch die
unterschiedliche Qualität der Belüftung der Anlage. Um eine freistehende
Anlage kann die Luft besser zirkulieren, dadurch erhitzt sich das Solarmodul
weniger stark. Die Temperatur eines Moduls ist ein wesentlicher Einflussfaktor
auf den Wirkungsgrad der ablaufenden Prozesse.

Der Parameter $G'$ lässt sich einfach durch $G'=\frac{G}{1000}$ beschreiben.

Die Koeffizienten $k_1,\dots,k_6$ sind abhängig davon, ob die PV-Anlage aus
kristallinen Silizium-Zellen oder aus Dünnschicht-Modulen besteht, wobei hier
die verwendeten Werkstoffe Einfluss haben. Diese Parameter wurden
experimentell bestimmt, sind aber leider nicht dokumentiert und können deshalb
nicht näher erläutert werden.

Die Kennzahl $l$ bezeichnet die witterungsunabhängigen Verluste, welche durch
Wechselrichter, weitere Elektronik und Leitungen hervorgerufen werden.

Die Globalstrahlung bezieht PVGIS durch Messungen aus Bodenstationen, welche
größtenteils durch Pyranometer durchgeführt werden. Da diese Messgeräte die
horizontal auf die Erde auftreffende Sonneneinstrahlung bestimmen, muss
$G(\alpha,\beta)$ aus diesen Messungen unter Berücksichtigung von $\alpha$ und
$\beta$ berechnet werden. Leider wird die Art der Umrechnung nicht näher
erläutert.

\section{Schlussbetrachtung}

Mit Hilfe der uns vorliegenden EEG-Stammdaten lässt sich keine Prognose über
die zu erwartende einspeisbare elektrische Leistung erstellen, da viele
benötigte Parameter zu den einzelnen Anlagen nicht vorliegen. Zusätzlich wird
dieses Vorhaben durch die komplexen physikalischen und meteorologischen
Prozesse, welche in der Atmosphäre und innerhalb der PV-Anlage ablaufen,
erschwert. Obwohl mit dem  PVGIS ein gut dokumentiertes System vorliegt,
reicht die Dokumentation dazu nicht aus, um die ablaufenden Berechnungen
wirklich vollständig nachvollziehen zu können. Wenn auch diese letzten
Unklarheiten beseitigt und entweder verpflichtend durch das EEG oder über eine
öffentliche Plattform auf freiwilliger Basis die Anlagenbetreiber mehr
Kennzahlen zu den einzelnen Anlagen zur Verfügung stellen würden, wäre
zumindest eine ungefähre Prognose über die zu erwartende Leistung
realisierbar.

\section{Quellen}
\begin{itemize}
\item \url{http://re.jrc.ec.europa.eu/pvgis/apps4/pvest.php}
\item \url{http://re.jrc.ec.europa.eu/pvgis/apps4/PVcalchelp_de.html}
\item
  \url{http://www.powerparc.de/uploads/media/Gutachten_Bad_Mergentheim.pdf} 
\end{itemize}
\end{document}
