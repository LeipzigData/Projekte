\documentclass[a4paper,11pt]{article}
\usepackage{a4wide,german,amsmath,url}
\usepackage[utf8]{inputenc}
\parindent0cm
\parskip3pt

\title{Bericht über Energiedaten im Internet\\ Wintersemester 2013/14}

\author{Andreas Krause}
\date{Januar 2014}

\begin{document}
\maketitle

\section{Einleitung}
Zur Verarbeitung von Energiedaten ist es erforderlich diese von einem Anbieter
zu bekommen. Dank des Erneuerbare-Energien-Gesetz sind Netzbetreiber und
Übertragungsnetzbetreiber verpflichtet, Daten von Anlagen zu veröffentlichen,
die Strom aus so genannten erneuerbaren Energien erzeugen. Erneuerbare
Energien haben den Ruf, besonders umweltfreundlich zu sein, da sie meist Strom
aus ``unerschöpflichen'' Quellen gewinnen, wie zum Beispiel Sonnenernergie
oder Windkraft.

Die veröffentlichten Informationen dieser Anlagen bestehen aus
Anlagestammdaten und Ab"|rechnungs- bzw.\ Bewegungsdaten. In Kapitel
\ref{anlagestammdaten} wird auf die Anlagestammdaten eingegangen, wobei die
Gemeinsamkeiten und Unterschiede der Daten aus verschiedenen Quellen
untersucht werden. Die Abrechnungsdaten werden anschließend in Kapitel
\ref{abbrechnungsdaten} behandelt. Es folgt eine kurze Zusammenfassung in
Kapitel \ref{summary}.

\section{Anlagestammdaten}
\label{anlagestammdaten}

\subsection{Quellen der Anlagedaten}
Nach dem Erneuerbare-Energien-Gesetz müssen Netzbetreiber Standort- und
Leistungsinformationen der Anlagen veröffentlichen. Diese Daten werden auch
Stammdaten genannt. Weiterhin bieten auch die Übertragungsnetzbetreiber
diese Informationen an. Im Falle der Anlagen im Leipziger Stadtgebiet kommen
also die hiesigen Netzbetreiber sowie der für diese Region zuständige
Übertragungsnetzbetreiber zur Beschaffung der Stammdaten in Frage. 

In Leipzig gibt es zwei Netzbetreiber, die Netz Leipzig GmbH \cite{netzle}
sowie die Mitteldeutsche Netzgesellschaft Strom mbH \cite{mitnetz}. Beide
Unternehmen bieten die Stammdaten als Download im PDF-Format an
\cite{netzle-stammdaten, mitnetz-stammdaten}.  Von der Homepage des
Übertragungsnetzbetreibers 50Hertz Transmission GmbH kann man sich die
Stammdaten der Anlagen im CSV-Format herunterladen \cite{50hertz}, wobei es
einige Filteroptionen gibt. Bei dem von uns verwendeten Datensatz wurden nur
Anlagen aus Sachsen verwendet. Eine weitere Quelle der Stammdaten ist die
Informationsplattform der deutschen Übertragungsnetzbetreiber,
\url{http://www.eeg-kwk.net}. Hier findet man zahlreiche Informationen zum
Erneuerbare-Energien-Gesetz und zur Kraft-Wärme-Kopplung. Es werden ebenfalls
die Anlagestammdaten aller Übertragungsnetzbetreiber in Deutschland im
CSV-Format zum Download angeboten \cite{eegkwk-stammdaten}.

Die in unserem Energieprojekt genutzten Daten der Anlagen stammen aus dem Jahr
2012 und wurden Mitte 2013 veröffentlicht. Lediglich die Daten, die von
50Hertz stammen, enthalten auch Anlagen aus dem Jahr 2013, da die Daten
ständig aktualisiert werden. Die im Projekt verwendete Version des
50Hetz-Datensatzes stammt vom November 2013. In der nachfolgenden Tabelle ist
eine Übersicht der Veröffentlichungsdaten und Aktualität der einzelnen
Datensätze zu finden.

\begin{table}[ht]
\centering
\begin{tabular}[h]{r|c|c}
Quelle & Veröffentlichungsdatum & Aktualität der Daten  \\\hline
50Hertz & November 2013  & November 2013  \\\hline
eeg-kwk.net & 31.05.2013 & 2012  \\\hline
MitNetz Strom mbH & 03.06.2013 & 2012 \\\hline
Netz Leipzig GmbH & 08.10.2013 & 2012
\end{tabular}
\caption{Aktualität der Anlagestammdaten}
\label{tab:anlageaktualitaet}
\end{table}

\subsection{Quantität der Anlagedaten}
Die Menge der Anlagedaten der verschiedenen Quellen variiert stark. Das liegt
daran, dass die Datensätze der Netzbetreiber nur Anlagen aus ihren Netzen
enthalten und folglich zueinander disjunkt sind. Die Daten von 50Hertz
hingegen enthalten alle Anlagen der Regelzone von 50Hertz und somit auch die
Anlagen der Netzbetreiber. Ebenso verhält es sich mit den Anlagestammdaten von
eeg-kwk.net. In Tabelle \ref{tab:anlagequantitaet} ist eine Auflistung der
Anzahl an Anlagen der einzelne Datensätze zu finden.

\begin{table}[ht]
\centering
\begin{tabular}[h]{r|c|c}
Quelle & Anlagen insgesamt & Anlagen in Leipzig \\\hline
50Hertz & 30217 & 1319 \\
(nur Anlagen aus Sachsen) & & (davon 122 von 2013) \\\hline
eeg-kwk.net & 117598 & 1217  \\\hline
MitNetz Strom mbH & 31190 & 579 \\\hline
Netz Leipzig GmbH & 638 & 612 
\end{tabular}
\caption{Anzahl der Anlagedaten verschiedener Quellen}
\label{tab:anlagequantitaet}
\end{table}

Die in Tabelle \ref{tab:anlagequantitaet} ersichtliche Differenz der Anzahl
Leipziger Anlagen zwischen den Datensätzen von 50Hertz und eeg-kwk.net kommt
zustande, da die Daten von 50Hertz über 100 Anlagen aus dem Jahr 2013
enthalten, die in den Daten von eeg-kwk.net nicht vorkommen, da diese aus dem
Jahr 2012 stammen. Trotz der Berücksichtigung dieses Umstandes stimmen die
Anzahlen aber trotzdem nicht überein. Auch die Summe der Leipziger Anlagen aus
den Daten der Netzbetreiber ist geringer als die in den anderen Datensätzen
vorhandene Anzahl Leipziger Anlagen, obwohl es nur diese zwei Netzbetreiber in
Leipzig gibt. Insgesamt gibt es 1335 Leipziger Anlagen, die sich in einem der
vier Datensätze befinden. Dadurch wird deutlich, dass es in den Datensätzen
Anlagen gibt, die in anderen nicht vorhanden sind, obwohl sie es sein müssten.
Ein Grund könnte sein, dass Anlagen aus den Datensätzen von 2012 mittlerweile
außer Betrieb genommen wurden und somit in dem 50Hertz-Datensatz nicht mehr
auftauchen. Eine weitere Möglichkeit besteht darin, dass die einzelnen
Datensätze unvollständig sind.

\subsection{Qualität der Anlagedaten}

Die Stammdaten der verschiedenen Quellen unterscheiden sich nicht nur in der
Menge der Anlagen, sondern auch darin, welche Informationen zu den Anlagen
enthalten sind. Dabei gibt es Anlageninformationen, die in allen Quellen
vorhanden sind und solche, die nur in einzelnen Datensätzen vorkommen. Tabelle
\ref{tab:attribs} gibt einen Überblick über die Attribute der einzelnen
Datensätze.

\begin{table}[ht]
\centering
\begin{tabular}[h]{r|c}
Datensatz & Attribute\\\hline
Gemeinsame Attribute & Anlagennummer, Ort, Strasse, Bundesland, \\
aller Datensätze & Nennleistung, Einspeisungsebene, Energieträger, \\
 & Datum der Inbetriebnahme  \\\hline
50Hertz.com & Netzbetreiber, \\
 & Biomasseanlagen: KWK-Anteil, Technologie \\\hline
eeg-kwk.net & Übertragungsnetzbetreiber, Netzbetreiber, \\
 & Netzbetreiber Betriebsnummer,leistungsgemessene Anlage, \\
 & Regelbarkeit, Zeitpunkt des Netzzuganges, Zeitpunkt \\
 & des Netzabganges, Zeitpunkt der Außerbetriebnahme \\\hline
MitNetz Strom mbH & Zeitreihentyp \\\hline
Netz Leipzig GmbH & Zeitpunkt der Außerbetriebnahme  \\
\end{tabular}
\caption{Attribute der Stammdaten}
\label{tab:attribs}
\end{table}

Es ist zu erkennen, dass in jedem Datensatz der Standort, die Leistung und die
Art der Energieerzeugung der Anlagen enthalten ist. Weiterhin besitzt jeder
Datensatz zusätzliche Attribute der Anlagen. So sind beispielsweise
zusätzliche Informationen zu Biomasseanlagen beim 50Hertz-Datensatz zu finden,
die bei den Datensätzen der Netzbetreiber fehlen. Einige zusätzliche Attribute
haben nur sehr selten Werte bei den Anlagen. Ein Beispiel hierzu ist der
Zeitpunkt der Außerbetriebnahme, da nahezu alle Anlagen in Betrieb sind.
 
Bei den Werten der Stammdaten müssen einige Besonderheiten beachtet werden. So
fehlt beispielsweise die vorgestellte „0“ bei den Postleitzahlen im
eeg-kwk.net Datensatz. Ebenfalls zu beachten ist, dass in den Datensätzen
teilweise die Ortsteile beim Ortsnamen mit angegeben sind.  

Da Flurstücke bei den Adressdaten angegeben sind, müssen diese eventuell
gesondert behandelt werden. In den Anlagen im Stadtgebiet von Leipzig kommen
solche Anlagen nicht vor. Wird allerdings das Gebiet auf das Leipziger Umland
oder eine andere Region ausgeweitet, so gehören auch Anlagen auf Flurstücken
zur relevanten Anlagenmenge.     

\subsubsection{Datenkonflikte}
Es hat sich gezeigt, dass sich die Anlagedaten teilweise widersprechen. Eine
automatisierte Analyse ergab 656 Konflikte zwischen den Stammdaten aller
Datensätze. Eine Verteilung dieser Konflikte bezüglich der Attribute ist in
Tabelle \ref{tab:konflikte} zu sehen.  

\begin{table}[ht]
\centering
\begin{tabular}[h]{r|c|c}
Attribut & Anzahl der Konflikte \\\hline
Leistung & 618   \\\hline
Energieträger & 18   \\\hline
Zeitpunkt der Inbetriebnahme & 10   \\\hline
PLZ & 5   \\\hline
Strasse & 3   \\\hline
Hausnummer & 2  
\end{tabular}
\caption{Verteilung der Datenkonflikte bezüglich Attribute}
\label{tab:konflikte}
\end{table}

Den größten Teil der Konflikte machen unterschiedliche Leistungsangaben aus.
Hierbei handelt es sich um Abweichungen, die auf unterschiedliche Genauigkeit
der Leistungswerte zurückzuführen sind. Diese Konflikte können beispielsweise
aufgelöst werden, indem bei der entsprechenden Anlage der Wert mit der
höchsten Präzision genommen wird.

Die meisten anderen Konflikte beruhen auf abweichenden Schreibweisen für den
gleichen Wert, bspw.\ „Wind“ und „Windenergie“ als Energieträger, oder
Schreibfehler. Bei diesen kann man ebenfalls einen ``richtigen Wert''
festlegen.

Lediglich 11 Konflikte können nicht ohne Weiteres aufgelöst werden. Hierbei
handelt es sich zum einen um einen falschen Eintrag als Energieträger, bei dem
einerseits „Solar“, andererseits „Bio“ als Wert in Frage kommt, sowie die 10
Konflikte bezüglich des Datums der Inbetriebnahme. Da bei diesen Konflikten
nicht entscheidbar ist, welcher Wert der richtige ist, müssen diese Daten als
fehlerhaft betrachtet werden. In der im Projekt erstellten Ontologie sind
dabei beide abweichenden Werte enthalten mit einem Hinweis in Form eines
\texttt{rdfs:comment}.

\section{Abrechnungsdaten}
\label{abbrechnungsdaten}
Die Netzbetreiber sind nach dem EEG ebenfalls verpflichtet, Jahresberichte zu
veröffentlichen, aus denen hervorgeht, wie viele Anlagen wie viel Strom
produziert haben und wie hoch die Vergütung war. Die Netz Leipzig GmbH bietet
diesen Bericht unter \cite{netzle-bericht} zum Download als PDF an. Der
Bericht der Mitteldeutschen Netzgesellschaft Strom mbH ist auf der Homepage
veröffentlicht \cite{mitnetz-bericht}. Bei diesen Berichten sind allerdings
die Daten nicht anlagebezogen, sondern eine Zusammenfassung aller Anlagen der
entsprechenden Netzbetreiber. Es lassen sich also keine Informationen zu den
einzelnen Anlagen gewinnen.

Die Abrechnungsdaten der einzelnen Anlagen können aber von 50Hertz bezogen
werden \cite{50hertz-bericht}. Hierbei können die schon von den
Anlagestammdaten bekannten Filter eingesetzt werden. Das Resultat sind die
Abbrechungsdaten jeder Anlage in einem bestimmten Jahr. In diesen Daten
befinden sich die Anlagestammdaten der Anlagen sowie weitere
Abbrechnungsinformationen wie produzierter EEG-Strom, die Vergütungskategorie,
die jeweilige Vergütung und vermiedene Netznutzungsentgelte. Es kann also
genau nachvollzogen werden, welche Anlage wie viel Strom in einem Jahr
produziert hat. Bei der Vergütung muss beachtet, dass die gelieferte
Energiemenge je nach Vergütungskategorie unterschiedlich vergütet wird.

\section{Zusammenfassung und Ausblick}
\label{summary}
Energiedaten von Erneuerbare-Energie-Anlagen können von den jeweiligen
Netzbetreibern bzw.\ vom Übertragungsnetzbetreiber erhalten werden. Diese
Daten werden entweder einmal im Jahr für das vorangegangene Jahr
veröffentlicht oder ständig aktualisiert (50Hertz). Die gelieferten
Informationen über die Anlagen bestehen zum einen aus Anlagestammdaten, die
hauptsächlich aus Standort-, Leistungs- und Energieerzeugungsinformationen
bestehen, zum anderen werden auch Abbrechnungsdaten veröffentlicht.

Die Anlagestammdaten sind zum Teil fehlerhaft, da sich Informationen aus
verschieden Quellen zu einigen Anlagen teilweise widersprechen. Es ist auch
nicht auszuschließen, dass die Menge der Anlagen bei bestimmten Quellen
unvollständig ist. Es gibt also einen gewissen Qualitätsmangel bei den
angebotenen Anlagestammdaten. 

Die Abrechnungsdaten, die von den Netzbetreibern veröffentlicht werden, sind
für alle Anlagen zusammengefasst und deswegen zu allgemein. Laut dem EEG
müssen diese aber auch nicht mehr Daten veröffentlichen, weshalb sich das in
Zukunft auch nicht ändern wird. Allerdings kann man zu einem gegebenen
Geschäftsjahr anlagenbezogene Abbrechnungsdaten von 50Hertz beziehen, wodurch
man genaue Aussagen über die reale Leistungsfähigkeit einer Anlage bekommt --
im Gegensatz zu der in den Anlagestammdaten angegebenen Nennleistung.
 
Da im Erneuerbare-Energien-Gesetz keine genauen Vorgaben existieren, welche
Daten in welcher Form veröffentlicht werden sollen, kann man nicht
ausschließen, dass sich Menge und Qualität der Anlagedaten verändert. So
könnte es aus datenschutzrechtlichen Gründen beispielsweise dazu kommen, dass
die Standortdaten eingeschränkt werden. Dies ist schon jetzt beim
Übertragunsnetzbetreiber TransnetBw der Fall, bei dem sich die Standortdaten
nur auf Ort und Postleitzahlen beschränken \cite{transnetbw}. Änderungen des
Erneuerbare-Energien-Gesetz könnten ebenfalls dazu führen, dass sich die Menge
der angebotenen Energiedaten ändert. In der Politik ist das
Erneuerbare-Energien-Gesetz ein aktuelles Streitthema, sodass dessen Änderung
in absehbarer Zeit durchaus denkbar wäre. Es ist daher nicht absehbar, wie
sich die Bereitstellung von Energiedaten in Zukunft entwickelt.

\bibliographystyle{plain}
\begin{thebibliography}{10}

\bibitem{50hertz} Homepage des Übertragungsnetzbetreibers 50Hertz, letzter
  Zugriff am 30.01.2014.  \newblock \url{http://www.50hertz.com}.

\bibitem{50hertz-bericht} Abbrechnungsdaten der Anlagen in der 50hertz
  Regelzone, letzter Zugriff am 31.01.2014.  \newblock
  \url{http://www.50hertz.com/cps/rde/xchg/trm_de/hs.xsl/166.htm}.

\bibitem{50hertz-stammdaten} {CSV}-{E}xport der {EEG}-stammdaten von
  50{H}ertz, letzter Zugriff am 30.01.2014.  \newblock
  \url{http://www.50hertz.com/cps/rde/xchg/trm_de/hs.xsl/165.htm}.

\bibitem{eegkwk} {I}nformationsportal der {Ü}bertragungsnetzbetreiber, letzter
  Zugriff am 30.01.2014.  \newblock \url{http://www.eeg-kwk.net}.

\bibitem{eegkwk-stammdaten} {D}ownload der {A}nlagestammdaten vom
  {I}nformationsportal der {Ü}bertragungsnetzbetreiber, letzter Zugriff am
  30.01.2014.  \newblock
  \url{http://www.eeg-kwk.net/de/Anlagenstammdaten.htm}.

\bibitem{mitnetz} {H}omepage des {N}etzbetreibers {M}itteldeutsche
  {N}etzgesellschaft {S}trom mb{H}, letzter Zugriff am 30.01.2014.  \newblock
  \url{https://www.mitnetz-strom.de}.

\bibitem{mitnetz-bericht} {J}ahresbericht des {J}ahres 2012 des
  {N}etzbetreibers {M}itteldeutsche {N}etzgesellschaft {S}trom mb{H}, letzter
  Zugriff am 31.01.2014.  \newblock
  \url{https://www.mitnetz-strom.de/Unternehmen/ZahlenFakten/ErneuerbareEnergien}.

\bibitem{mitnetz-stammdaten} {A}nlagestammdaten des {J}ahres 2012 des
  {N}etzbetreibers {M}itteldeutsche {N}etzgesellschaft {S}trom mb{H}, letzter
  Zugriff am 30.01.2014.  \newblock
  \url{https://www.mitnetz-strom.de/irj/go/km/docs/z_ep_em_unt_documents/em/mitnetzstrom/Dokumente/Stammdaten_EEG-Jahresmeldung_2012_MITNETZ%20STROM.pdf}.

\bibitem{netzle} {H}omepage des {N}etzbetreibers {N}etz {L}eipzig {G}mb{H},
  letzter Zugriff am 30.01.2014.  \newblock \url{http://www.netz-leipzig.de}.

\bibitem{netzle-bericht} {J}ahresbericht des {J}ahres 2012 des
  {N}etzbetreibers {N}etz {L}eipzig {G}mb{H}, letzter Zugriff am 31.01.2014.
  \newblock
  \url{http://www.netz-leipzig.de/netzanschluesse/erzeugungsanlagen/#c81}.

\bibitem{netzle-stammdaten} {A}nlagestammdaten des {J}ahres 2012 des
  {N}etzbetreibers {N}etz {L}eipzig {G}mb{H}, letzter Zugriff am 30.01.2014.
  \newblock
  \url{http://www.netz-leipzig.de/netzanschluesse/erzeugungsanlagen/?eID=dam_frontend_push&docID=358}.

\bibitem{transnetbw} {A}uswertung der {EEG}-{A}nlagenstammdaten für
  {P}hotovoltaik-{A}nlagen: keine {V}erbesserung in {S}icht, letzter Zugriff
  am 31.01.2014.  \newblock \url{http://hsi-solar-invest.com/?p=5364}.

\end{thebibliography}

\end{document}

